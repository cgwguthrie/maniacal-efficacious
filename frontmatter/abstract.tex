The era of Big Data is upon us bringing with it a range of new challenges, and encouraging the formulation of new approaches for cleaning, processing and using these enormous amounts of data. These new methods have led to the development of a range of technologies designed to meet the needs of Big Data.

This project focuses on a subset of the new technologies, in particular those products designed to deliver high performance querying of large data sets.  It shall compare leading NOSQL solutions (e.g., MongoDB [1], and Neo4j [2]) against modern search and analytics engines (e.g., ElasticSearch [3], and SOLR [4]).  The overall goal is to compare and contrast the functionally, performance and analytical capabilities of the different solutions.  With the ultimate aim of gaining an understanding of, and insight into, these technologies and their application to Big Data.
	
The student will produce several versions of a prototype application; with each version employing a different technology (or approach). This work will be undertaken using a real world dataset from the biological environment.\\[2em]

\noindent [1] http://www.mongodb.org/about/introduction/

\noindent [2] http://www.neo4j.org/

\noindent [3] http://www.elasticsearch.org/webinars/introduction-elk-stack/ 

\noindent [4] http://lucene.apache.org/solr/