The era of Big Data is upon us, bringing with it a range of new challenges ``Without big data, you are blind and deaf and in the middle of a freeway"\cite{moore}. The significance of these challenges encouraged the formulation of new approaches for cleansing, processing and utilising enormous amounts of data.

Data is being collated and stored every second of every day and the value of doing so has never been greater. Billion dollar companies such as Google and Amazon dominate the market in data collection and pride themselves in knowing everything about everything. Former CEO of Google, Eric Schmidt famously said in 2010 ``We know where you are. We know where you've been. We can more or less know what you're thinking" \cite{schmidt}. Thus the power of data collection has led to the development of a range of technologies designed to meet the needs of big data.

The purpose of the following research, by way of investigation, is to deliver an insightful examination of a subset of new technologies which deliver high performance querying of large datasets. The ultimate aim of the project is to gain an understanding of these technologies and achieve a level of mastery which permits a thorough scrutiny of their application to big data.

There are a number of different indexing solutions available. In order to encapsulate a comprehensive examination a focus will be on leading NoSQL solutions, modern search and analytics engines and for comparative reasons a conventional relational database management system. The following technologies which will be used for the project: 

\begin{itemize}
\item Relational database (Section \ref{mysql})
\item Document-orientated database (Section \ref{mongo})
\item Graph database (Section \ref{neo})
\item Wide column database (Section \ref{cassandra})
\end{itemize}

\section{Objectives}\label{objectives}
The three key objectives and main intended outcomes for the project are:
\begin{enumerate}
\item Investigate the strengths and weaknesses of the functionality each technology provides.
\item Compare and contrast the analytical capabilities of each technology by way of querying prototype models.
\item Conduct a comparative analysis to investigate the scalability of each technology.
\end{enumerate}

\section{Project Motivation}
My interest in the field of data science stems from university modules I have undertaken as part of my BSc Computer Science degree. Modules such as `Database Management Systems', `Data Mining and Machine Learning' and a course I am currently studying `Big Data'. The material involved in these courses have given me an insight into the field of data science and provided me with the opportunity to get a hands on feel for the manipulation, cleansing and processing of a variety of real life data sets and database systems.

Whilst studying for my degree, I have successfully completed modules which have required a working knowledge of MySQL as a prerequisite therefore my comprehension of MySQL is proficient. One of the main attractions to undertaking this project was to be given the opportunity to learn about a number of next generation database management systems. It was important for me to undertake a project in which I will be able to apply my learning and findings to progress in a career path within the data science field.

\section{Definitions}
The data source being used in this document comes from the biological field and therefore relies on an understanding of basic concepts and terms. The below table of definitions provides an overview into the main terms used throughout the report with the aim of providing the reader of the document with a sufficient level of understanding. The table also includes the definitions of generally obscure terms and phrases which will be discussed throughout this project.

\begin{center}
    \begin{tabular}{ |p{0.3\linewidth} | p{0.7\linewidth} |}
    \hline
    \textbf{Term} & \textbf{Definition} \\ \hline
    Edinburgh Mouse Atlas Project (EMAP) & The combined research projects of Dr Duncan Davidson and Prof Richard Baldock. \\ \hline
    EMAP anatomy & A freely available, structured, stage specific list of 13,000+ terms that describe visible anatomical structures in the developing mouse embryo. \\ \hline
    Edinburgh Mouse Atlas Project Abstract (EMAPA) & A refined and algorithmically developed non-stage specific anatomical ontology. representation of the EMAP anatomy. \\ \hline
    Edinburgh Mouse Atlas of Gene Expression (EMAGE) & A database of in situ gene expression data in the developing mouse embryo.  \\ \hline
    Theiler Stage (TS) & Each stages defines the development of a mouse embryo by a set of organism structure criteria. \\ \hline
    DNA & Deoxyribonucleic acid. Molecule which carries genetic instructions. \\ \hline
    RNA & Ribonucleic acid. An acid which is present in all living cells.  \\ \hline
    Assay & One or more assay comprises an experiment. \\ \hline
    Gene & A hereditary unit consisting of a sequence of DNA \cite{emap}. \\ \hline
    Gene expression & The specific activity of a gene when a segment of DNA is copied into RNA. \\ \hline
    Specimen &  A sample of something. For example, an animal or a plant or a piece of human tissue.\\ \hline
    In situ & Latin for `in place'. This term refers to the original position of the anatomy when being experimented on. \\ \hline
    Probe & Used to detect DNA and RNA on membranes and preparing for in situ experiments. \\ \hline
    Not only SQL (NoSQL) & A non-relational database environment which is useful for very large sets of distributed data. Allows rapid, ad-hoc organisation and analysis of extremely high-volume, disparate data types. \\ \hline
        Ontology & Refers to the science of describing the kinds of entities in the world and how they are related. \\ \hline
    \end{tabular}
\end{center}
