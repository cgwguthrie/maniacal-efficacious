\chapter{Dataset Origins}
The dataset used in the prototype applications is a real world dataset taken from the biological environment. The data is constructed by an ontology derived from the combined research projects undertaken on the e-Mouse Atlas Project (EMAP) by Dr Duncan Davidson and Professor Richard Baldock.\begin{wrapfigure}{r}{0.25\textwidth}\includegraphics[width=0.9\linewidth]{images/ema_logo}\end{wrapfigure} The name EMAP carries a certain amount of ambiguity as it is the name of the project that developed the anatomy, and is also the name of the anatomy itself. Therefore with the motivation of clarity, I will refer to the project that developed the anatomy as e-Mouse Atlas (EMA) and the name of the anaotmy as EMAP. Inspired by the findings of Theiler (1989) and Kaufman (1992), EMA uses embryological mouse models to provide a detailed map of mouse development. The EMAP has a developed collection of three dimensional computer models of mouse embryos at the consecutive stages of growth generation with anatomical domians joined by an ontology of anatomical names. The main deliverable of the EMA resource is to provide a comprehensive visualisation of the postimplantation of mouse development and to induct an investiagtion of the gene expression in the postimplantation mouse embryo.

The EMA ontology has several different branches of deliverables, each providing an alternative aspect of the evolution of a mouse embryo. The branches which will be utilised for this research project are the timed stage specific structures which are EMAP and the aggregated non stage specific e-mouse Atlas Project Abstract (EMAPA) which are resepectively discussed below. The EMA dataset's were chosen as the source of data for this research as it is a freely available, rich and substantial data source.

\section{EMAP}
The devised EMAP ontology was originally developed to deliver a structured and controlled vocabulary of stage-specific anatomical structures for the developing laboratory mouse. As the EMA research has progressed, the ontology has followed suit, continually under development and the current ontology is in scope for a forthcoming release.
\subsection{Data}
\subsection{Purpose}
\section{EMAPA}
\subsection{Data}
\subsection{Purpose}
