\externaldocument{big_data}
\chapter{Introduction}
The era of Big Data is upon us bringing with it a range of new challenges "Without big data, you are blind and deaf and in the middle of a freeway." (Moore, 2012) The importance which accompany these challenges encouraged the formulation of new approaches for cleaning, processing and using these enormous amounts of data. (Section \ref{bigdata})

Data is being collated and stored every second of every day and the value of doing so has never been greater. Billion dollar companies such as Google and Amazon dominate the market in data collection and pride themselves in knowing everything about everything. Former CEO of Google, Eric Schmidt famously said in 2010 "We know where you are. We know where you've been. We can more or less know what you're thinking..." (Schmidt, 2010) Thus the power of data collection has led to the development of a range of technologies designed to meet the needs of Big Data.

The purpose of the following research, by way of investigation, is to deliver an insightful examination of a subset of new technologies which deliver high performance querying of large datasets. The ultimate aim of the research is to gain an understanding of these technologies and achieve a level of mastery which permits a thorough scrutiny of their application to Big Data.

There are a number of different indexing solutions available however for the means of this project to encapsulate a comprehensive examination a focus will be on leading NoSQL solutions, modern search and analytics engines and for comparative reasons a conventional relational database management system. The following technologies to be used for the project: 
\begin{enumerate}
\item MongoDB - Document-Oriented Database 
\item Neo4j - Graph Database
\item Apache Cassandra - Distributed Database
\item MySQL - Relational Database Management System
\end{enumerate}

\section{Objectives}
The key objectives and main intended outcomes for the project are:
\begin{itemize}
\item Investigate the strengths and weaknesses of the functionality each technology provides.
\item Compare and contrast the analytical capabilities of each technology by way querying prototype models.
\item Conduct a comparative analysis to investigate the scalability of each technology.
\end{itemize}

\section{Project Intentions}
Choosing an indexing solution for querying Big Data sets can be difficult as there are so many options with limits on each respective functionalities. Taking the A \& O into consideration the project intends to identify the best overall performing indexing solution for this dataset 
http://www.dataversity.net/nosql-or-rdbms-choosing-the-right-database-for-you/