\chapter{Importing Data}\label{dataload}
As discussed in section \ref{etlprocess}, the load stage of an ETL procedure can often become the most time consuming phase of the pipeline. This chapter describes the implemented procedures and examines the functionality each solution provides to complete the load process.

\section{Put the data in databases}\label{loadsection}
The final stage in the data modelling process was to import the EMAGE dataset into the created data models. For each of the database systems, a different approach was required to be undertaken. To ensure a balanced and impartial evaluation, each of the datasets were converted into a CSV file format and manipulated by the functional tools the respective systems provide.

\subsection{MySQL - data load}
There are a couple of main ways in which data can be loaded into a MySQL database. You can manually insert the data, row by row in the MySQL shell command prompt, using an \textbf{INSERT INTO} statement. However, to implement a full database using this method would be extremely time consuming and laborious. While time may not be of the essence in certain circumstances, manually writing 200,000 rows of insert statements is in no way the optimal solution to complete this task. An alternative option available is to use the \textbf{mysqlimport} command. The mysqlimport client is simply a command-line interface to the \textbf{LOAD DATA INFILE} statement. For readers unfamiliar with this statement, code snippet \ref{code:ldi} represents an example LOAD DATA INFILE implementation.
\newpage
\begin{lstlisting}[language=SQL, caption=Example LOAD DATA INFILE statement., label=code:ldi]
mysql > LOAD DATA INFILE '/home/callum/emageData/assay.csv'
	 	 -> INTO TABLE assays
		 -> FIELDS TERMINATED BY ','
		 -> LINES TERMINATED BY '\n'
		 -> (emageID, probeID, type);
\end{lstlisting}

The mysqlimport command can take a number of parameters some of which include, delete (empty the table before import), lock (lock all tables for writing before processing any text files) and force (continue even if an SQL error occurs). While these are noteworthy and useful functions, they are not required in this instance. The mysqlimport statement used to load the EMAGE dataset into MySQL can be found in code snippet \ref{code:mysqlload}.
\begin{lstlisting}[language=SQL, caption=Example LOAD DATA INFILE statement., label=code:mysqlload]
---
---Import data into all tables in one command
---
mysqlimport -u root -p --local --fields-terminated-by=, emage assays.csv publications.csv sources.csv specimens.csv stages.csv textannotations.csv genes.csv anatomystructures.csv

\end{lstlisting}








