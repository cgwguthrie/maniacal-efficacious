\chapter{Knowledge Representation Languages}
A Knowledge Representation (KR) language can be defined as: for any given interpretation of a sentence or string of text the KR must have the ability to effectively and unambiguously express knowledge in a both a human and computer manageable form. 

There are a number of options and possibilities for communicating data and information which range from binary representation to meta markup languages such as Extensible Markup Language (XML) for example. Markup languages which are easily read by humans such as XML which as a result of its rigid set of rules lends its self to both humans and machines. Comparatively binary notation which uses 1's and 0's to represent data, while relatively cheap in terms of computing power the ability to comprehend this notation requires a unique and specific skill set.

\subsection{Semantic Web}\label{semanticweb}
The Semantic Web is an extension of the Web through standards by the World Wide Web Consortium (W3C). ``The standards promote common data formats and exchange protocols on the Web, most fundamentally the Resource Description Framework (RDF)." \cite{semantic}

The Semantic Web has two main intended outcomes. The first is about the standardised formats of data pulled from variety of sources, whereas the original Web concentrated on the interchange of documents \cite{semantic}. The second outcome of the Semantic Web is the language for recording the relation between data and objects in the real world. ``That allows a person, or a machine, to start off in one database, and then move through an unending set of databases which are connected not by wires but by being about the same thing."  \cite{semantic}. 

\subsection{Web Ontology Language OWL}\label{owl}
The Web Ontology Language OWL is a language representation standard for designing and authoring Web ontologies produced from the World Wide Web Consortium W3C. \cite{owl}. The OWL file format is designed to be used by applications required to process the content of the information and to be humanly readable. ``[OWL] is intended to provide a language that can be used to describe the classes and relations between them that are inherent in Web documents and applications."\cite{owl}. The OWL languages are characterised by formal semantics and are built upon the W3C standard RDF format - discussed in section \ref{etltool}

\subsection{OBO}\label{obo}
The OBO flat file format is an ontology representation language. ``The concepts it models represent a subset of the concepts in the OWL description logic language, with several extensions for meta-data modelling and the modelling of concepts that are not supported in DL languages." \cite{obo}

\cite{obo} outlines the intended outcome of the file format aiming to achieve the following criteria :
\begin{itemize}
\item Human readability
\item Ease of parsing
\item Extensibility
\item Minimal redundancy
\end{itemize}