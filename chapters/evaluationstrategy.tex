\chapter{Evaluation Strategy}
The purpose of this chapter is to discuss the methods used to carry out the primary research for this project. In order to achieve the objectives outlined in section \ref{objectives} a formal set of requirements have been established to meet each intended target. Section \ref{requirements} details how each objective will be evaluated and examples are provided of how each technology will be measured by way of competency questions.

\section{Requirements}\label{requirements}

The first objective of the project is to investigate the strengths and weaknesses of the functionality each technology provides. Section \ref{techeval} of this document discusses each technology which will be evaluated in this project and their respective intended functionalities. 

Compare and contrast the analytical capabilities of each technology.

\begin{enumerate}
\item E
\end{enumerate}

The final key objective to be evaluated in this project is to conduct a comparative analysis of the scalability of each technology. This will be achieved by loading the data sources in to each prototype data model. The specific requirements which will measure the performance of each database are :

\begin{enumerate}
\item Ease of ETL implementation - The focus of this requirement will be to evaluate any challenges which were faced during the ETL process.
\item
\item
\end{enumerate}
\subsection{EMAPA Competency Questions}\label{competency}

Competency Questions for EMAPA:
\begin{itemize}
\item For structure X, find all the ancestors.
\item For structure X, find all the decedents.
\item For structure X, find all structures in the same group (this only makes sense in limited situations, e.g., find all "hair").
\item Find all the structures in Theiler Stage X.
\item Find all the structures in Theiler Stages X to Y (e.g., 17-19).
\item What stages does structure X appear in?
\item Given search term X return the "best match" structure (e.g., if Isearch on "heart" I would expect the output to be heart, heart atrium, heart septum  heart mesentery).
\item Given an EMAPA ID return the name of the structure.
\item Given a name return the EMAPA ID.
\end{itemize}
%If you also have the EMAP data, you may wish to include:
%Given an EMAPA ID retrieve the EMAP ID.
% Given an EMAP ID retrieve the EMAPA ID.

\subsection{EMAGE Competency Questions}
Where is gene X expressed?
What is expressed in structure X?
Which genes are expressed in both structures X and Y?  (This is
so-called co-expression).
Which genes are most commonly co-expressed? (This is getting towards
BI and is possibly out with the scope of your thesis, but it does no
harm to discuss it.)
