\chapter{Requirements}\label{evaluationstrategy}
The purpose of this chapter is to discuss the project requirements used to carry out the primary research for the project. In order to achieve the objectives outlined in section \ref{objectives} a formal set of requirements have been established to meet each intended target. Section \ref{requirements} details how each objective will be achieved.



\section{Objective and Requirement Strategies}\label{requirements}
\parindent 0pt
\subsubsection*{Objective 1}
\textbf{Evaluate the strengths, weaknesses and limitations of each query language.}

The first objective it to investigate the strengths and weaknesses of querying functionality each technology provides. In order to evaluate the functional limitations of each technology, prototype data models will be developed for each solution. The solutions will then be loaded with the EMAPA and EMAGE datasets.

I have created a number of competency based questions. These questions will be translated in to the relevant query language for each of the datasets. The difficulty of the queries range from a basic level, where it is expected that all of the solutions will return the intended output, to an advanced complexity level. The difficulty of writing the queries will be evaluated to assess the overall performance of the database solution.

\subsubsection*{Objective 2}
\textbf{Compare and contrast the analytical capabilities of each technology.}

This will follow on from the tasks undertaken in the first objective. This objective will also identify and analyse the difficulties which may or may not have arisen in the first objective on the ease of writing the query.

\begin{itemize}
\item If the query is possible in a given database how rich a return of the dataset can it provide.
\item Does any database offer any querying capabilities/functionality which compared with another which aids the query in any way.
\item Extra features which the system offers which are useful for analysis and visualisation.
\end{itemize}

\subsubsection*{Objective 3}
\textbf{Assess the scalability of each of the database solutions.}

The final key objective to be evaluated in this project is to conduct a comparative analysis of the scalability of each technology. This will be achieved by loading the data sources in to each prototype data model. The specific requirements which will measure the performance of each database are:

\begin{itemize}
\item Ease of ETL implementation - The focus of this requirement will be to evaluate any challenges which were faced during the ETL process.
\item Did the data model handle the volume of data and were there any issues as a result.
\end{itemize}

\newpage
\begin{table}[H]
\centering
\resizebox{\textwidth}{!}{
\begin{tabular}{|c|l|l|c|}
\hline
\textbf{Query Number} & \multicolumn{1}{c|}{\textbf{English}} & \multicolumn{1}{c|}{\textbf{Expected return}} & \textbf{Rating} \\ \hline
1 & All structures at Theiler Stage X. & Theiler Stage and Structure ID. & 1 \\ \hline
2 & \begin{tabular}[c]{@{}l@{}}All structures between Theiler\\ Stage X and Y.\end{tabular} & \begin{tabular}[c]{@{}l@{}}Structure ID, Theiler Stage X and Theiler Stage\\Y.\end{tabular} & 1 \\ \hline
3 & Where is Gene X expressed? & \begin{tabular}[c]{@{}l@{}}Name of the gene, the structure where the\\ gene was found and the EMAGE ID where the\\ gene was found.\end{tabular} & 2 \\ \hline
4 & What is expressed in structure X? & \begin{tabular}[c]{@{}l@{}}Name of the gene(s) found in the structure. The\\structure ID and the name of the structure.\\ The Theiler Stage(s) of the structure. The\\EMAGE ID of the structure.\end{tabular} & 3 \\ \hline
5 & \begin{tabular}[c]{@{}l@{}}Which genes are stored in\\ structures X and Y?\end{tabular} & \begin{tabular}[c]{@{}l@{}}Name and ID of the gene(s). The ID of\\structure X and ID of structure Y.\end{tabular} & 4 \\ \hline
6 & \begin{tabular}[c]{@{}l@{}}Which Genes are most\\ commonly co-expressed?\end{tabular} & \begin{tabular}[c]{@{}l@{}}Name of the gene(s) and the count of unique\\ structures the gene is expressed in.\end{tabular} & 5 \\ \hline
7 & Calculate transitive closure. & The name of each structure and its parent. & 5 \\ \hline
\end{tabular}}
\caption{Competency queries for each database system. Rating scale: 1 (simple - complex) 5}
\label{tab:competency}
\end{table}

\parindent 15pt
