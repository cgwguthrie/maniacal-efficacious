\chapter{Technical Discussion}
The focus of this technical investigation was to develop my knowledge of the NoSQL and indexing solutions examined in the project and to gain insight into the subject matter by reflecting on previously conducted research. Prior to this research, my understanding of the functionality provided by Neo4j and Apache Cassandra was minimal. Throughout my university degree I have undertaken modules which have stipulated a working knowledge of MySQL as a prerequisite therefore my comprehension of MySQL is proficient. - INCOMPLETE

\section{Database Classification}
One of the first decisions to be made when when selecting a database is the characteristics of the data you are looking to leverage. (Dash, 2013) There are a multitude of options available with many different classifications. - EXPAND

\subsection{Document-Oriented Database}
Document-orientated database (DODB) are designed for storing, retrieving and managing document files such as XML, JSON and BSON. The documents stored in a DODB model are data objects which describe the data in the document, as well as the data itself. DODBs are 
\subsubsection{MongoDB}\label{mongo}
The premise for using a DODB such as MongoDB is simplicity, speed and scalability. The DODB data model is unrestrictive and flexible in nature which lends itself to programmers. Its popularity stems from the ability to query on all fields and the instinctive mapping of the data model to objects in modern programming languages. (MongoDB White Paper, 2015) Within this free flowing environment documents can become as sophisticated and complex as required; information about a document record can be sub categorised by the integration of nested data.

\subsection{Graph Database}
In computing, a graph database is a database that uses graph structures for semantic queries with nodes, edges and properties to represent and store data. Graph databases employ nodes, properties, and edges.

Nodes represent entities such as people, businesses, accounts, or any other item you might want to keep track of. Properties are pertinent information that relate to nodes. For instance, if "Wikipedia" were one of the nodes, one might have it tied to properties such as "website", "reference material", or "word that starts with the letter 'w'", depending on which aspects of "Wikipedia" are pertinent to the particular database. Edges are the lines that connect nodes to nodes or nodes to properties and they represent the relationship between the two. Most of the important information is really stored in the edges. Meaningful patterns emerge when one examines the connections and interconnections of nodes, properties, and edges
\subsubsection{Neo4j}\label{neo}

\subsection{Distributed Database}
A distributed database is a database in which storage devices are not all attached to a common processing unit such as the CPU, controlled by a distributed database management system (together sometimes called a distributed database system).
\subsubsection{Apache Cassandra}\label{cassandra}

\subsection{Relational Database}
A relational database is a collection of data items organized as a set of formally-described tables from which data can be accessed or reassembled in many different ways without having to reorganize the database tables. The relational database was invented by E. F. Codd at IBM in 1970.
\subsubsection{MySQL}\label{mysql}

\section{Previous Research - Change Heading}
NoSQL has become a bit of a "...ubiquitous and possibly overused term." (Lerman, 2011)
\subsection{TBC}